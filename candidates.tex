\documentclass[a4paper,10pt]{article}
\usepackage[utf8]{inputenc}

\usepackage[utf8]{inputenc}
\usepackage{amsfonts}
\usepackage{amsmath,amssymb,amsfonts,graphicx}
\usepackage[english]{babel}
\usepackage{natbib}
\usepackage{color}
\usepackage{subfigure}
\usepackage{graphicx}
\usepackage{rotating}
\usepackage{longtable}
\usepackage{lscape}
\usepackage{hyperref}
\usepackage{textcomp}
\usepackage[T1]{fontenc}


%opening
\title{Reporte auto-generado por \textit{alertreport}: procesado alerta aLIGO}
\author{\textbf{alertreport} package @IATE - CONICET - UNC}

\begin{document}

\maketitle

\begin{abstract}
Reporte simple mostrando caracter\'{i}sticas de la alerta y objetos de observaci\'on 
candidatos a ser anfitriones de un evento de Onda gravitacional.
\end{abstract}

\section{El mapa celeste de aLIGO: valor de likelihood y objetos asociados}
El mapa de probabilidad (\textit{likelihood} en ingl\'es) en el cielo se muestra en el diagrama
de la figura \ref{fig:aitoff_likelihood_map}. El mismo exhibe en mapa de calor las zonas donde es 
m\'as probable que el evento se encuentre localizado.

\begin{figure}
 \centering
 \includegraphics[width=0.7\textwidth]{./plots/allsky_likelihoodmap.png}
 % likelihood_map_aitoff2.png: 0x0 pixel, 300dpi, 0.00x0.00 cm, bb=
 \caption{Mapa de probabilidades de aLIGO, en proyecci\'on ``aitoff'' para todo el cielo.
 En rojo se puede ver la posicion media del cenith de Mac\'on durante la noche de la alerta.}
 \label{fig:aitoff_likelihood_map}
\end{figure}

El cruce de estas zonas con el cielo observable desde Mac\'on, se desprende la figura 
\ref{fig:gnomom_view_Macon_likelihoodmap} la cual presenta una vista ``cenital'' de el 
mapa de calor de probabilidad de aLIGO observable en la localizaci\'on geogr\'afica del cerro.
Este mapa muestra entonces que cielo accedemos, y en la barra inferior muestra el nivel de 
\textit{likelihood} en colores.

\begin{figure}
 \centering
 \includegraphics[width=0.7\textwidth]{./plots/gnomom_view_Macon_likelihoodmap.png}
 % candidates_aitoff.png: 0x0 pixel, 300dpi, 0.00x0.00 cm, bb=
 \caption{Proyecci\'on ``cenital'' desde Mac\'on, del mapa de calor de probabilidad de aLIGO.}
 \label{fig:gnomom_view_Macon_likelihoodmap}
\end{figure}



\begin{landscape}
\begin{small}
\include{./plots/targets_textable}
\end{small}

\end{landscape}

The RA-Dec diagram for the candidates is in fig \ref{fig:candidatesRA-Dec}
\begin{figure}
 \centering
 \includegraphics[width=0.7\textwidth]{./plots/aitoff_selected_targets.png}
 % candidates_radec.png: 0x0 pixel, 300dpi, 0.00x0.00 cm, bb=
 \caption{Ra-Dec projection of candidates in the sky.}
 \label{fig:candidatesRA-Dec}
\end{figure}



\end{document}
